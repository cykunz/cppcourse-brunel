\hypertarget{index_intro_sec}{}\section{Introduction}\label{index_intro_sec}
This project has the final goal of simulation a network of 12500 neurons. However, it is composed of several steps in order to reach that final goal.

The first step was to implement the \hyperlink{classNeuron}{Neuron} class.

Then, two neurons had to be created and a connection installed between them, meaning whenever one spiked, the other is supposed to receive a signal straight away. Note that only the first neuron received an input from the outside, therefore the second neuron's potential stayed at 0 until a signal was received. When the signal was received, the second neuron's membrane potential will decrease exponentially.

Next, a certain delay was implemented, as the second neuron is not supposed to receive the signal straight away. This delay was implemented with the help of a ring buffer\-: a vector of a fixed size (delay\-\_\-steps+1), keeping track of all spikes received at a certain time step. Note that this buffer was implemented with modulos in this step.

Finally, the idea was to create a \hyperlink{classNetwork}{Network} class, having 12500 neurons. Within those neurons, 10000 were excitatory and 2500 were inhibitory. For simplicity reasons, the first 10000 neurons in my network were excitatory, and the rest inhibitory. Each neuron must receive 10\% input\-: meaning it has to have 1250 connections with other neurons in the network, from which 1000 are excitatory connections and 250 inhibitory. Please note that this buffer, in my project, was implemented in the following way\-: in order to get rid of modulos, slowing down my program, the class \hyperlink{classNetwork}{Network} contains two indexes which iterate on each neuron's buffer at each time step\-: one index corresponds to the time of the network, and the second one takes into consideration the delay\-: it starts at the index delay\-\_\-steps. This way, the two indexes are always delay\-\_\-steps apart from each other. 